
\documentclass[12pt]{article}
\usepackage{geometry}
 \geometry{letterpaper,left=25mm,top=25mm,right=25mm}
\usepackage[utf8]{inputenc}
\usepackage[spanish]{babel} %Poner algunas palabras reservadas en español
\usepackage{authblk} %Poner instituto en la portada
%Paquetes para símbolos matematicos
\usepackage{amsmath}
\usepackage{amsthm}
\usepackage{amssymb}
\usepackage{bbm}
\usepackage[]{algorithm2e} %Paquete para algoritmos
\usepackage{enumerate}
%Paquete para imagenes
\usepackage{graphicx}
\graphicspath{{img/}}
\newtheorem*{thm}{Teorema}

\title{Tarea 1}
\author{Fernando Márquez Pérez \\ Juan Antonio Jasso Oviedo \\ Emiliano Dom\'inguez Cruz}
\date{06/09/2019}
\affil{Facultad de Ciencias\\UNAM}

\begin{document}
\begin{titlepage}
    \maketitle
\end{titlepage}

%EJERCIO 8
8.Calcular f'(x) para cada una de las siguientes funciones (sin importar los dominios de f(x) y de f'(x ))
%EJERCIO A
\item $\mathbf{f(x)=sin(x²+x)}$
%EJERCIO B
\item $\mathbf{f(x)=sin(x)+sin(x²)}$
%EJERCIO C
\item $\mathbf{f(X)=sin(cos(x))}$
    \begin{align*}

        Recordamos por la regla de la cadena primero derivamos sin() y posterormente lo multiplicamos por la derivada de cos(x)\\
        Tenemos\\
        g(x)=sin(cos(x))\\
        g'(x)=cos(cos(x)) \ \ \  &&\text{Por el teorema que probamos en clase de f(x)=sin(x) entonces f'(x)=cos(x)}\\
        Observamos\\
        h(x)=cos(x)  \ \ \ &&\text{Por el teorema que probramos en clase de f(x)=cos(x) entonces f'(x)=-sin(x}\\
        h'(x)=-sin(x)\\
        es decir\\
        f'(x)=(cos(cos(x)))(-sin(x)) \ \ \ &&\text{Por la regla de la cadena se multiplican}\\

    \end{align*}

%EJERCIO D
\item $\mathbf{f(x)=sin(sin(x))}$
%EJERCIO E
\item $\mathbf{f(x)=sin(x+sin(x))}$
%EJERCIO F
\item $\mathbf{f(x)=sin(cos(sin(x)))}$
    \begin{align*}
        Recordamos por la regla de la cadena primero derivamos sin(cos(sin(x))) y posteriormente multiplicamos por la derivada de cos(sin(x)) y posteriormente por sin(x) \\
        Tenemos,\\
        g(x)=sin(cos(sin(x)))\\
        g'(x)=cos(cos(sin(x))) \ \ \ &&\text{Por el teorema que probamos en clase}\\
        Además, tambíen tenemos,\\
        h(x)=cos(sin(x))\\
        h'(x)=-sin(sin(x)) \ \ \ &&\text{Por el teorema que probamos en clase}\\
        Posteriormente\\
        t(x)=sin(x)\\
        t'(x)=cos(x) \ \ \ &&\text{Por el teorema que probamos en clase}\\
        es decir\\
        f'(x)=cos(cos(sin(x)))*-sin(sin(x))*cos(x) \ \ \ &&\text{Por la regla de la cadena se multiplican}\\
    \end{align*}

%EJERCIO G
\item $\mathbf{f(x)=sin(cos(x)/x)}$
%EJERCIO H
\item $\mathbf{f(x)=(sin(cos(x)))/x}$
%EJERCIO I
\item $\mathbf{f(x)=cos(cos(x))/x}$
    \begin{align*}
        Observamos\\
        f(x)=g(x)/h(x)  \ \ \ &&\text{con g(x)=cos(cos(x)) y h(x)=x}\\
        Recordamos\\
        f'(x)=((h(x)*g'(x))-(g(x)*h'(x)))/g(x)² \ \ \ &&\text{por el teorema que probamos en clase}\\
        Observamos\\
        h'(x)=1 \ \ \ &&\text{}\\
        y\\
        g'(x)=sin(cos(x))(-sin(x)) \ \ \ &&\text{por la regla de la cadena como en ejercicios pasados derivamos de afuera hacia adentro y luego lo multiplicamos}\\
        es decir\\
        f'(x)= (((x)*(sin(cos(x))*(-sin(x))))-(cos(cos(x)))(1))/x² \ \ \ &&\text{Por regla de la cadena y el teorema de la derivada de el producto de dos funciones.}\\
    \end{align*}

\end{document}
