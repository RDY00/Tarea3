
\documentclass[12pt]{article}
\usepackage{geometry}
 \geometry{letterpaper,left=25mm,top=25mm,right=25mm}
\usepackage[utf8]{inputenc}
\usepackage[spanish]{babel} %Poner algunas palabras reservadas en español
\usepackage{authblk} %Poner instituto en la portada
%Paquetes para símbolos matematicos
\usepackage{amsmath}
\usepackage{amsthm}
\usepackage{amssymb}
\usepackage{bbm}
\usepackage[]{algorithm2e} %Paquete para algoritmos
\usepackage{enumerate}
%Paquete para imagenes
\usepackage{graphicx}
\graphicspath{{img/}}
\newtheorem*{thm}{Teorema}
\setlength{\parindent}{0cm}

\title{Tarea 1}
\author{Fernando Márquez Pérez \\ Juan Antonio Jasso Oviedo \\ Emiliano Dom\'inguez Cruz}
\date{06/09/2019}
\affil{Facultad de Ciencias\\UNAM}

\begin{document}
\begin{titlepage}
    \maketitle
\end{titlepage}

1. Determine si las siguientes funciones son continuas en \(x_0\).

\begin{enumerate}[\hspace{12px} a)]
    \item
    \( f(x)=
    \begin{cases}
        \sqrt{x^2-1}\text{, si} \ x \geq 1\\
        x^2-2x+1\text{, si} \ x \in [0,1]
    \end{cases}
    \)
    en $x_0 = 1$\\
    \item
    \( h(x)=
    \begin{cases}
        \frac{|x|}{x}\text{, si} \ x \neq 0\\
        1\text{, si} \ x=0\\
    \end{cases}
    \)
    en $x_0=0$
    \item
    \( g(x)=
    \begin{cases}
        \sqrt{1-x^2}\text{, si} \ x \in [0,1]\\
        -\sqrt{1-(x-2)^2}\text{, si} \ x \in [1,2]
    \end{cases}
    \)
    en $x_0=1$
\end{enumerate}

2. Se inyecta un fármaco a un paciente cada 12 horas. En la Fig. 1 se muestra la concentración $c(t)$ del fármaco en el torrente sanguieo después de $t$ horas.

\begin{enumerate}[\hspace{12px} a)]
    \item ¿Para qué valores de $t$, $c(t)$ tiene discontinuidades?
    \item ¿Qué tipo de discontinuidades tiene?
\end{enumerate}

3. Mostrar que existe algún número $x$, tal que:

\begin{enumerate}[\hspace{12px} a)]
    \item \(\sin x = x-1\)
    \item \(x^{179}+\displaystyle\frac{163}{1+x^2+\sin^2 x}=119\)
    \item \(\cos x - \displaystyle\frac{1}{2}=x-1\)
    \item \((2x^2-2)^2=-x+1\)
\end{enumerate}

4. Vea si en los siguientes incisos se cumple el teorema de valor intermedio y, en ese caso, calcula un valor intermedio.

\begin{enumerate}[\hspace{12px} a)]
    \item \(f(x)=x^3\) en $[-1,1]$
    \item \(g(x)=x^3\) en $[0,2]$
    \item \(h(x)=x^2+4x+4\) en $[0,1]$
    \item \(k(x)=3x^2-x-1\) en $[-1,1]$
\end{enumerate}

5. Pruebe que las ecuaciones dadas, tienen una raíz en el intervalo que se señala.

\begin{enumerate}[\hspace{12px} a)]
    \item \(x^3+7x^2-3x-5=0\) en $[-3.2,0.1]$
    \item \(x^5-4x^3+x^2-1=0\) en $[-2.1,1.5]$
    \item \(x\sin x-\displaystyle\frac{1}{2}=0\) en $[-1,2]$
    \item \(x\cos x+\displaystyle\frac{1}{2}=0\) en $[-1,3.5]$
\end{enumerate}

6. Partiendo de la definición de derivada, mostrar que:

\begin{enumerate}[\hspace{12px} a)]
    \item si \(f(x)=\displaystyle\frac{1}{x}\), entonces \(f'(a)=-\displaystyle\frac{1}{a^2}\) para \(a \neq 0\)
    \item si \(f(x)=\displaystyle\frac{1}{x^2}\), entonces \(f'(a)=-\displaystyle\frac{2}{a^3}\) para \(a \neq 0\)
    \item si \(f(x)=\sqrt{x}\), entonces \(f'(a)=-\displaystyle\frac{1}{2\sqrt{a}}\) para \(a > 0\)
\end{enumerate}

7. Encontrar la ecuación de la recta tangente en el punto \((a,f(a))\) para las siguiente funciones.

\begin{enumerate}[\hspace{12px} a)]
    \item \(f(x)=\displaystyle\frac{1}{x}\) para \(a \neq 0\)
    \item \(f(x)=\displaystyle\frac{1}{x^2}\) para \(a \neq 0\)
    \item \(f(x)=\sqrt{x}\) para \(a > 0\)
\end{enumerate}

8. Calcular \(f'(x)\) para cada una de las siguientes funciones (sin importar los dominios de \(f\) y \(f'\)).

\begin{enumerate}[\hspace{12px} a)]
    \item \(f(x) = \sin(x+x^2)\)
    \item \(f(x) = \sin(x) + \sin(x^2)\)
    \item \(f(x) = \sin(\cos(x))\)
    \item \(f(x) = \sin(\sin(x))\)
    \item \(f(x) = \sin(x+\sin(x))\)
    \item \(f(x) = \sin(\cos(\sin(x)))\)
    \item \(f(x) = \sin\left(\displaystyle\frac{\cos(x)}{x}\right)\)
    \item \(f(x) = \displaystyle\frac{\sin(\cos(x))}{x}\)
    \item \(f(x) = \displaystyle\frac{\cos(\cos(x))}{x}\)
\end{enumerate}


\begin{enumerate}[\hspace{12px} a)]
    \item \(\)
\end{enumerate}

\vfill

%EJERCIO 8
8.Calcular $f'(x)$ para cada una de las siguientes funciones (sin importar los dominios de $f(x)$ y de $f'(x)$).
%EJERCIO A
%    \begin{align*}
        Recordamos por la regla de la cadena primero derivamos sin() y posterormente lo multiplicamos por la derivada de cos(x)\\
        Tenemos\\
        g(x)=sin(cos(x))\\
        g'(x)=cos(cos(x)) \ \ \  %&&\text{Por el teorema que probamos en clase de f(x)=sin(x) entonces f'(x)=cos(x)}\\
        Observamos\\
        h(x)=cos(x)  \ \ \ %&&\text{Por el teorema que probramos en clase de f(x)=cos(x) entonces f'(x)=-sin(x}\\
        h'(x)=-sin(x)\\
        es decir\\
        f'(x)=(cos(cos(x)))(-sin(x)) \ \ \ %&&\text{Por la regla de la cadena se multiplican}\\

%    \end{align*}


%EJERCIO F
%    \begin{align*}
        Recordamos por la regla de la cadena primero derivamos sin(cos(sin(x))) y posteriormente multiplicamos por la derivada de cos(sin(x)) y posteriormente por sin(x) \\
        Tenemos,\\
        g(x)=sin(cos(sin(x)))\\
        g'(x)=cos(cos(sin(x))) \ \ \ %&&\text{Por el teorema que probamos en clase}\\
        Además, tambíen tenemos,\\
        h(x)=cos(sin(x))\\
        h'(x)=-sin(sin(x)) \ \ \ %&&\text{Por el teorema que probamos en clase}\\
        Posteriormente\\
        t(x)=sin(x)\\
        t'(x)=cos(x) \ \ \ %&&\text{Por el teorema que probamos en clase}\\
        es decir\\
        f'(x)=cos(cos(sin(x)))*-sin(sin(x))*cos(x) \ \ \ %&&\text{Por la regla de la cadena se multiplican}\\
%    \end{align*}

%EJERCIO I

%    \begin{align*}
        Observamos\\
        f(x)=g(x)/h(x)  \ \ \ %&&\text{con g(x)=cos(cos(x)) y h(x)=x}\\
        Recordamos\\
        $f'(x)=((h(x)*g'(x))-(g(x)*h'(x)))/g(x)^2$ \ \ \ %&&\text{por el teorema que probamos en clase}\\
        Observamos\\
        h'(x)=1 \ \ \ %&&\text{}\\
        y\\
        $g'(x)=sin(cos(x))(-sin(x))$ \ \ \ %&&\text{por la regla de la cadena como en ejercicios pasados derivamos de afuera hacia adentro y luego lo multiplicamos}\\
        es decir\\
        $f'(x)= (((x)*(sin(cos(x))*(-sin(x))))-(cos(cos(x)))(1))/x^2$ \ \ \ %&&\text{Por regla de la cadena y el teorema de la derivada de el producto de dos funciones.}\\
%    \end{align*}

\end{document}
