
\documentclass[12pt]{article}
\usepackage{geometry}
 \geometry{letterpaper,left=25mm,top=25mm,right=25mm}
\usepackage[utf8]{inputenc}
\usepackage[spanish]{babel} %Poner algunas palabras reservadas en español
\usepackage{authblk} %Poner instituto en la portada
%Paquetes para símbolos matematicos
\usepackage{amsmath}
\usepackage{amsthm}
\usepackage{amssymb}
\usepackage{bbm}
\usepackage[]{algorithm2e} %Paquete para algoritmos
\usepackage{enumerate}
%Paquete para imagenes
\usepackage{graphicx}
\graphicspath{{img/}}
\newtheorem*{thm}{Teorema}
\setlength{\parindent}{0cm}

\title{Tarea 1}
\author{Fernando Márquez Pérez \\ Juan Antonio Jasso Oviedo \\ Emiliano Dom\'inguez Cruz}
\date{06/09/2019}
\affil{Facultad de Ciencias\\UNAM}

\begin{document}
\begin{titlepage}
    \maketitle
\end{titlepage}

1. Determine si las siguientes funciones son continuas en \(x_0\).

\begin{enumerate}[\hspace{12px} a)]
    \item
    \( f(x)=
    \begin{cases}
        \sqrt{x^2-1}\text{, si} \ x \geq 1\\
        x^2-2x+1\text{, si} \ x \in [0,1]
    \end{cases}
    \)
    en $x_0 = 1$\\
    \item
    \( h(x)=
    \begin{cases}
        \frac{|x|}{x}\text{, si} \ x \neq 0\\
        1\text{, si} \ x=0\\
    \end{cases}
    \)
    en $x_0=0$
    \item
    \( g(x)=
    \begin{cases}
        \sqrt{1-x^2}\text{, si} \ x \in [0,1]\\
        -\sqrt{1-(x-2)^2}\text{, si} \ x \in [1,2]
    \end{cases}
    \)
    en $x_0=1$
\end{enumerate}

\vfill

%EJERCIO 8
8.Calcular $f'(x)$ para cada una de las siguientes funciones (sin importar los dominios de $f(x)$ y de $f'(x)$).
%EJERCIO A
%    \begin{align*}
        Recordamos por la regla de la cadena primero derivamos sin() y posterormente lo multiplicamos por la derivada de cos(x)\\
        Tenemos\\
        g(x)=sin(cos(x))\\
        g'(x)=cos(cos(x)) \ \ \  %&&\text{Por el teorema que probamos en clase de f(x)=sin(x) entonces f'(x)=cos(x)}\\
        Observamos\\
        h(x)=cos(x)  \ \ \ %&&\text{Por el teorema que probramos en clase de f(x)=cos(x) entonces f'(x)=-sin(x}\\
        h'(x)=-sin(x)\\
        es decir\\
        f'(x)=(cos(cos(x)))(-sin(x)) \ \ \ %&&\text{Por la regla de la cadena se multiplican}\\

%    \end{align*}


%EJERCIO F
%    \begin{align*}
        Recordamos por la regla de la cadena primero derivamos sin(cos(sin(x))) y posteriormente multiplicamos por la derivada de cos(sin(x)) y posteriormente por sin(x) \\
        Tenemos,\\
        g(x)=sin(cos(sin(x)))\\
        g'(x)=cos(cos(sin(x))) \ \ \ %&&\text{Por el teorema que probamos en clase}\\
        Además, tambíen tenemos,\\
        h(x)=cos(sin(x))\\
        h'(x)=-sin(sin(x)) \ \ \ %&&\text{Por el teorema que probamos en clase}\\
        Posteriormente\\
        t(x)=sin(x)\\
        t'(x)=cos(x) \ \ \ %&&\text{Por el teorema que probamos en clase}\\
        es decir\\
        f'(x)=cos(cos(sin(x)))*-sin(sin(x))*cos(x) \ \ \ %&&\text{Por la regla de la cadena se multiplican}\\
%    \end{align*}

%EJERCIO I

%    \begin{align*}
        Observamos\\
        f(x)=g(x)/h(x)  \ \ \ %&&\text{con g(x)=cos(cos(x)) y h(x)=x}\\
        Recordamos\\
        $f'(x)=((h(x)*g'(x))-(g(x)*h'(x)))/g(x)^2$ \ \ \ %&&\text{por el teorema que probamos en clase}\\
        Observamos\\
        h'(x)=1 \ \ \ %&&\text{}\\
        y\\
        $g'(x)=sin(cos(x))(-sin(x))$ \ \ \ %&&\text{por la regla de la cadena como en ejercicios pasados derivamos de afuera hacia adentro y luego lo multiplicamos}\\
        es decir\\
        $f'(x)= (((x)*(sin(cos(x))*(-sin(x))))-(cos(cos(x)))(1))/x^2$ \ \ \ %&&\text{Por regla de la cadena y el teorema de la derivada de el producto de dos funciones.}\\
%    \end{align*}

\end{document}
