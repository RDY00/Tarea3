
\documentclass[12pt]{article}
\usepackage{geometry}
 \geometry{letterpaper,left=25mm,top=25mm,right=25mm}
\usepackage[utf8]{inputenc}
\usepackage[spanish]{babel} %Poner algunas palabras reservadas en español
\usepackage{authblk} %Poner instituto en la portada
%Paquetes para símbolos matematicos
\usepackage{amsmath}
\usepackage{mathtools}
\usepackage{amsthm}
\usepackage{amssymb}
\usepackage{bbm}
\usepackage[]{algorithm2e} %Paquete para algoritmos
\usepackage{enumerate}
%Paquete para imagenes
\usepackage{graphicx}
\graphicspath{{img/}}
%Paquete para teoremas
\newtheorem*{thm}{Teorema}
%Quitar la sangria
\setlength{\parindent}{0cm}

\title{Tarea 3}
\author{Fernando Márquez Pérez \\ Juan Antonio Jasso Oviedo \\ Emiliano Dom\'inguez Cruz}
\date{08/11/2019}
\affil{Facultad de Ciencias\\UNAM}

\begin{document}
\begin{titlepage}
    \maketitle
\end{titlepage}

%EJERCICIO 1 ----------------------------------------------------------------------------------
1. Sea $b>0$. Usar el criterio de integrabilidad de $f$ sobre $[a,b]$, para mostrar que:

\begin{enumerate}[\hspace{9px} a)]
    %EJERCICIO A 
    \item \(\displaystyle\int_{0}^{b}\frac{x}{3}dx=\frac{b^2}{6}\)
    
    %EJERCICIO B
    \item \(\displaystyle\int_{0}^{b}\frac{x^2}{2}dx=\frac{b^3}{6}\)
    
    %EJERCICIO C 
    \item \(\displaystyle\int_{0}^{b}3x^2dx=b^3\)
    
    %EJERCICIO D
    \item \(\displaystyle\int_{0}^{b}x^3dx=\frac{b^4}{4}\)
    
\end{enumerate}

%EJERCICIO 2 ----------------------------------------------------------------------------------
2. Explica por qu\'e la funci\'on 
\(f(n)=
\begin{cases}
    \displaystyle\frac{1}{x} \quad \text{si} \ \ 0<x<1\\
    0 \quad \text{si} \ \ x=0
\end{cases}
\) no es integrable.

%EJERCICIO 3 ----------------------------------------------------------------------------------
3. Obtenga la cota de la forma \(m(b-a) \leq \displaystyle\int_{0}^{b}f(x)dx \leq M(b-a)\), en los intervalos indicados de los siguientes incisos.

\begin{enumerate}[\hspace{9px} a)]
    %EJERCICIO A
    \item \(\sin(x) \text{, en} \ -\pi \leq x \leq \pi\)
    
    %EJERCICIO B
    \item \(x^4 \ \text{en} \ -4 \leq x \leq 4\)
    
    %EJERCICIO C
    \item \(\tan(x) \ \text{, en} \ -\frac{\pi}{4} \leq x \leq \frac{\pi}{4}\)
    
\end{enumerate}

%EJERCICIO 4 ----------------------------------------------------------------------------------
4.
\begin{enumerate}[\hspace{9px} a)]
    %EJERCICIO A
    \item Mostrar que si $f$ es integrable sobre $[a,b]$ y \(f(x) \geq 0 \ \forall \ x \in [a,b]\), entonces \(\displaystyle\int_{a}^{b}f \geq 0\).
    
    %EJERCICIO B
    \item Mostrar que si $f$ y $g$ son integrables sobre $[a,b]$ y \(f(x) \geq g(x) \ \forall \ x \in [a,b]\), entonces \(\displaystyle\int_{a}^{b}f \geq \displaystyle\int_{a}^{b}g\).

\end{enumerate}

%EJERCICIO 5 ----------------------------------------------------------------------------------
5. Mostrar que: \quad \(\displaystyle\int_{ca}^{cb}f(t)dt=c\displaystyle\int_{a}^{b}f(ct)dt\)

%EJERCICIO 6 ----------------------------------------------------------------------------------
6. Sea $b>0$. Supongase que $f$ es una funci\'on integrable sobre $[-b,b]$.

\begin{enumerate}[\hspace{9px} a)]
    %EJERCICIO A
    \item Si $f$ es una funci\'on par, demostrar que \(\displaystyle\int_{-b}^{b}f(t)dt=2\int_{0}^{b}f(t)dt\).
    
    %EJERCICIO B
    \item Si $f$ es una funci\'on impar, demostrar que \(\displaystyle\int_{-b}^{b}f(t)dt=0\).

\end{enumerate}

%EJERCICIO 7 ----------------------------------------------------------------------------------
7. Hallas las areas de las regiones limitadas por:

\begin{enumerate}[\hspace{9px} a)]
    %EJERCICIO A 
    \item Las gr\'aficas de \(f(x)=x^2\) y \(g(x)=\displaystyle\frac{x^2}{2}+2\).
    
    %EJERCICIO B
    \item Las gr\'aficas de \(f(x)=x^2\) y \(g(x)=1-x^2\).
    
    %EJERCICIO C
    \item Las gr\'aficas de \(f(x)=x^2\) y \(g(x)=1-x^2\) y \(h(x)=2\).
    
    %EJERCICIO D
    \item Las gr\'aficas de \(f(x)=x^2\) y \(g(x)=x^2-2x+4\) y el eje vertical.

\end{enumerate}

%EJERCICIO 8 ----------------------------------------------------------------------------------
8. Hallar la derivada de cada una de las siguientes funciones.

\begin{enumerate}[\hspace{9px} a)]
    %EJERCICIO A
    \item \(F(x)=\displaystyle\int_{a}^{x^3}\sin^3(t)dt\)
    
    %EJERCICIO B
    \item \(F(x)=\displaystyle\int_{3}^{\left(\displaystyle\int_{1}^{x}\sin^3(t)dt\right)}\frac{1}{1+\sin^6(t)+t^2}dt\)
    
    %EJERCICIO C
    \item \(F(x)=\displaystyle\int_{15}^{x}\left(\int_{8}^{y}\frac{1}{1+\sin^2(t)+t^2}dt\right)dy\)

\end{enumerate}

%EJERCICIO 9 ----------------------------------------------------------------------------------
9. Para cda una de las siguiente integrales impropias. Mostrar si es convergente o no seg\'un sea el caso.

\begin{enumerate}[\hspace{9px} a)]
    %EJERCICIO A
    \item \(\displaystyle\int_{0}^{\infty}x^rdx\) si $r<-1$
    
    %EJERCICIO B
    \item \(\displaystyle\int_{1}^{\infty}\frac{1}{x}dx\)
    
    %EJERCICIO C
    \item \(\displaystyle\int_{0}^{a}x^r\) si $-1<r<0$
    
    %EJERCICIO D
    \item \(\displaystyle\int_{0}^{1}\frac{1}{x}dx\)

\end{enumerate}

%EJERCICIO 10 ----------------------------------------------------------------------------------
10. Muestre que la regi\'on \(A=\{(x,y) \ | \ x<1, \ 0 \leq y \leq \frac{1}{x}\}\) tiene \'area infinita.

%EJERCICIO 11 ----------------------------------------------------------------------------------
11. Encontrar la gr\'afica de las siguientes funciones.

\begin{enumerate}[\hspace{9px} a)]
    %EJERCICIO A 
    \item \(f(x)=\tan(x)-x\)
    
    %EJERCICIO B
    \item \(f(x)=x+\sin(x)\)
    
    %EJERCICIO C
    \item \(f(x)=\sin(x)+\sin(2x)\)

\end{enumerate}

%EJERCICIO 12 ----------------------------------------------------------------------------------
12. 
\begin{enumerate}[\hspace{9px} a)]
    %EJERCICIO A
    \item Partiendo de la f\'ormula para $\cos(2x)$, deducir las f\'ormulas para $\sin^2(x)$ y $\cos^2(x)$ en t\'erminos de $\cos(2x)$.
    
    %EJERCICIO B
    \item Mostrar que \(\cos\left(\displaystyle\frac{x}{2}\right)=\sqrt{\frac{1+\cos(x)}{2}}\) y \(\cos\left(\displaystyle\frac{x}{2}\right)=\sqrt{\frac{1-\cos(x)}{2}}\) para \(0 \leq x \leq \displaystyle\frac{\pi}{2}\).
    
    %EJERCICIO C
    \item Usar el primer inciso para calcular \(\displaystyle\int_{a}^{b}\sin^2(x)dx\) y \(\displaystyle\int_{a}^{b}\cos^2(x)dx\).

\end{enumerate}

%EJERCICIO 13 ----------------------------------------------------------------------------------
13. Mostrar que:

\begin{enumerate}[\hspace{9px} a)]
    %EJERCICIO A
    \item \(\arcsin'(x)=\displaystyle\frac{1}{\sqrt{1-x^2}}\), si \(-1<x<1\)
    
    %EJERCICIO B
    \item \(\arccos'(x)=\displaystyle-\frac{1}{\sqrt{1-x^2}}\), si \(-1<x<1\)
    
    %EJERCICIO C
    \item \(\arctan'(x)=\displaystyle\frac{1}{1+x^2}\), para toda $x$.

\end{enumerate}

%EJERCICIO 14 ----------------------------------------------------------------------------------
14. Aplicar la derivaci\'on logar\'itmica para obtener la derivada $f'(x)$ de cada una de las siguientes funciones:

\begin{enumerate}[\hspace{9px} a)]
    %EJERCICIO A
    \item \(f(x)=(1+x)(1+e^{x^2})\)
    
    %EJERCICIO B
    \item \(f(x)=(\cos(x))^{\sin(x)}+(\sin(x))^{\cos(x)}\)
    
    %EJERCICIO C
    \item \(f(x)=\displaystyle\frac{(3-x)^{1/3}x^3}{(1-x)(3+x)^{2/3}}\)
    
    %EJERCICIO D
    \item \(f(x)=\displaystyle\frac{e^x-e^{-x}}{e^{3x}(1+x^3)}\)
    
\end{enumerate}

%EJERCICIO 15 ----------------------------------------------------------------------------------
Hallar la gr\'afica de las siguientes funciones.

\begin{enumerate}[\hspace{9px} a)]
    %EJERCICIO A
    \item \(f(x)=e^{1+x}\)
    
    %EJERCICIO B
    \item \(f(x)=e^{\sin(x)}\)
    
    %EJERCICIO C
    \item \(f(x)=e^x+e^{-x}\)
    
    %EJERCICIO D
    \item \(f(x)=e^x-e^{-x}\)
    
    %EJERCICIO E
    \item \(f(x)=\displaystyle\frac{e^x-e^{-x}}{e^x+e^{-x}}=\frac{2}{e^{2x}+1}\)

\end{enumerate}

%EJERCICIO 16 ----------------------------------------------------------------------------------
16. Las siguientes funciones reciben el nombre de seno hiperb\'olico, coseno hiperb\'olico y tangente hiperb\'olica, respectivamente.

\begin{equation*}
    \sinh(x)=\frac{e^x-e^{-x}}{2}, \ \cosh(x)=\frac{e^x+e^{-x}}{2} \ y \ \tanh(x)=\frac{e^x-e^{-x}}{e^x+e^{-x}}
\end{equation*}

Mostrar que:

\begin{enumerate}[\hspace{9px} a)]
    %EJERCICIO A
    \item \(\cosh^2(x)-\sinh^2(x)=1\)
    
    %EJERCICIO B
    \item \(\sinh'(x)=\cosh(x)\)
    
    %EJERCICIO C
    \item \(\cosh'(x)=\sinh(x)\)
    
    %EJERCICIO D
    \item \(\tanh^2(x)+\displaystyle\frac{1}{\cosh^2(x)}=1\)
    
    %EJERCICIO E
    \item \(\tanh'(x)=\displaystyle\frac{1}{\cosh^2(x)}\)
    
    %EJERCICIO F
    \item \(\sinh(x+y)=\sinh(x)\cosh(y)+\sinh(y)\cosh(x)\)
    
    %EJERCICIO G
    \item \(\cosh(x+y)=\cosh(x)\cosh(y)+\sinh(y)\sinh(x)\)

\end{enumerate}

%EJERCICIO 17 ----------------------------------------------------------------------------------
17. Las funciones hiperb\'olicas inversas se pueden expresar por medio del logaritmo. Muestre que:

\begin{enumerate}[\hspace{9px} a)]
    %EJERCICIO A
    \item \(f(x)=x+\displaystyle\frac{1}{x}\)
    
    %EJERCICIO B
    \item \(f(x)=x+\displaystyle\frac{3}{x^2}\)
    
    %EJERCICIO C
    \item \(f(x)=\displaystyle\frac{x^2}{x^2-1}\)
    
    %EJERCICIO D
    \item \(f(x)=\displaystyle\frac{1}{x^2+1}\)

\end{enumerate}

%EJERCICIO 18 ----------------------------------------------------------------------------------
18. Evalua los l\'imites usando la regla de L'H\^opital.

\begin{enumerate}[\hspace{9px} a)]
    %EJERCICIO A 
    \item \(\displaystyle\lim_{x \to 0}\frac{e^x-1-x-\frac{x^2}{2}}{x^2}\)
    
    %EJERCICIO B
    \item \(\displaystyle\lim_{x \to 0}\frac{e^x-1-x-\frac{x^2}{2}-\frac{x^3}{6}}{x^3}\)
    
    %EJERCICIO C
    \item \(\displaystyle\lim_{x \to 0}\frac{e^x-1-x-\frac{x^2}{2}}{x^3}\)
    
    %EJERCICIO D
    \item \(\displaystyle\lim_{x \to 0}\frac{\ln(1+x)-x-\frac{x^2}{2}}{x^2}\)
    
    %EJERCICIO E
    \item \(\displaystyle\lim_{x \to 0}\frac{\ln(1+x)-x-\frac{x^2}{2}}{x^3}\)

\end{enumerate}

%EJERCICIO 19 ----------------------------------------------------------------------------------
19. Sin usar la regla de L'H\^opital, hallar los siguientes l\'imites.

\begin{enumerate}[\hspace{9px} a)]
    %EJERCICIO A
    \item \(\displaystyle\lim_{y \to 0}\frac{\ln(1+y)}{y}\)
    
    %EJERCICIO B
    \item \(\displaystyle\lim_{x \to \infty}x\ln\left(1+\frac{1}{x}\right)\)
    
    %EJERCICIO C
    \item \(e=\displaystyle\lim_{x \to \infty}\left(1+\frac{1}{x}\right)^x\)
    
    %EJERCICIO D
    \item \(e^a=\displaystyle\lim_{x \to \infty}\left(1+\frac{a}{x}\right)^x\)
    
\end{enumerate}

%EJERCICIO 20 ----------------------------------------------------------------------------------
20. Use el hecho que la poblaci\'on mundial en 1950 era de 2560 millones y 3040 millones en 1960 y un modelo de crecimiento poblacional, para responder las siguientes preguntas.

\begin{enumerate}[\hspace{9px} a)]
    %EJERCICIO A
    \item ¿Cuál es la tasa de crecimiento relativa?
    
    %EJERCICIO B
    \item Use el modelo de crecimiento poblacional para estimar cuánta gente habrá en 2020.
    
    %EJERCICIO C
    \item Indague y diga si ese número se acerca a la población actual. Cite la fuente.

\end{enumerate}

%EJERCICIO 21 ----------------------------------------------------------------------------------
21. La vida media del radio-226 es de 1590 años.

\begin{enumerate}[\hspace{9px} a)]
    %EJERCICIO A
    \item Una muestra e radio-226 tiene una masa de $100mg$. Encontrar una fórmula para la masa de la muestra que queda después de $t$ años.
    
    %EJERCICIO B
    \item Diga cuál es la masa después de 1000 años, dé su resultado en $mg$.
    
    %EJERCICIO C
    \item En qué tiempo la masa de la muestra se reducirá a $30mg$.

\end{enumerate}

\end{document}